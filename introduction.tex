The LSST covers a wide swath of unique science cases and observation types. Rubin Observatory will enable many scientific discoveries, especially target-of-opportunity (ToO) observations during the early commissioning period. 

Rubin Observatory is uniquely positioned to lead ToO observations through the 2020's and beyond due to it's unique technical capabilities. The $\textasciitilde10\deg^2$ field-of-view of the Rubin optical system allow ToO observations to survey a wide area, while the single-visit depth of the LSST-Camera allows single observations to observe the southern sky for faint transient phenomena. The combination of the large FOV and deep observations make Rubin Observatory an ideal tool for discovery of ToO phenomena.

The Rubin ToO program encompasses 3\% of the LSST, and includes GW, high-energy neutrino, potentially hazardous asteroids, and other time-sensitive astrophysical phenomena as different targets. Each target has a different observing strategy based on observing conditions, the conditions of the astrophysical event, and other parameters. The observing strategies are the product of community input, and were revised in 2024 (\cite{RubinToO2024}). These recommendations were accepted by the survey cadence optimization committee in January 2025 (\cite{PSTN-056}).

While other components of the LSST are not limited by the specific time of observation, ToO is uniquely in that the confirmation of a ToO counterpart requires rapid observations, ranging from mere minutes of alert receipt to hours. The different nature of observations requires different workflows, communication channels, and operations to ensure that ToO observations are valuable to the LSST. 

In the forthcoming sections, we describe the state of the Rubin ToO system (section \ref{sec:sysOverview}), the workflow during the science verification period (section \ref{sec:workflow}), the responses to mock and real alerts (section \ref{sec:ToOEvents}), and lessons learned from the early commissioning and science verification period (section \ref{sec:LessonsLearned}).